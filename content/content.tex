%! TEX program = lualatex lualatex lualatex

\section{Voraussetzungen für einen Antrag}
Möchtest du Geld für eine Sache, die der Studierendenschaft zu Gute kommt? Dann kannst du einen Antrag erstellen.

Wichtig! Der Antrag und das Vorhaben müssen \textbf{innerhalb einer Legislatur} (von Anfang April des aktuellen Jahres bis Ende März des folgenden Jahres) liegen.
Geht es bei deiner Sache um eine Angelegenheit (z.B. einen Vertrag) dann musst du diesen Antrag an das StuPa (Studierenden Parlament) stellen.  Wir gehen bei dieser Dokumentation nicht davon aus.

Damit über einen Antrag abgestimmt werden kann, muss dieser entweder von dir zwei Tage vor der Sitzung eingereicht werden oder du stellst dich und deinen Antrag in der Vorstandssitzung vor.



\section{Wo findet man die Unterlagen?}
Wenn du einen Antrag für die VoSi (Vorstandssitzung) erstellen möchtest, musst du zunächst die Vorlage aus der Cloud herunterladen. 
Die Vorlagen findet man in der GremienCloud (\href{https://cloud.asta-hsrm.de}{cloud.asta-hsrm.de}) unter:
\[\texttt{AStA Alle}\Rightarrow \texttt{3\_Anträge}\]

Dort kannst du sowohl eine Word-Datei (\href{https://cloud.asta-hsrm.de/s/foBwqpT6s7S7Xtc/download}{cloud.asta-hsrm.de/s/foBwqpT6s7S7Xtc/download}) 
als auch eine PDF-Datei (\href{https://cloud.asta-hsrm.de/s/NpPc5Y97R5dbSAT}{cloud.asta-hsrm.de/s/NpPc5Y97R5dbSAT}) finden. 

\section{Wie wird der Antrag richtig ausgefüllt?}
Nun widmen wir uns der Frage, was man alles benötigt, damit ein Antrag richtig ausgefüllt wird.

Als erstes füllst du in dem Feld \textit{Antragsteller}, was du auch unten abgebildet siehst, deine Kontaktdaten ein. 

Bei \texttt{Tätigkeit/Referat} schreibst entweder das Referat oder deine Tätigkeit hin, über welches dieser Antrag gestellt wird, zum Beispiel \textit{IT-Referat} oder \textit{Rechnungsprüfungsausschuss}.
Hast du hingegen vor, als Student:in mit anderen Studierenden etwas durchzuführen und benötigst dafür Geld, dann kannst du hier das Feld mit \textit{Student} ausfüllen.
\begin{figure}[ht!]
  \centering
  \includegraphics[width=\linewidth]{fig/deine_daten.png}
\end{figure}

Im nächsten Feld (\texttt{Hiermit wird ein Antrag gestellt für:}) erläuterst du kurz für was du den Antrag stellst. 

Im Feld \texttt{Begründung und Kosten} begründe bitte warum du das Geld dafür brauchst und wie andere Studierende davon profitieren. 
Außerdem kannst du auch begründen wie es zu dem Betrag kommt, denn du benötigst.

Als letztes fühlst du das Feld \texttt{Kosten belaufen sich auf} mit der Summe aus, die du benötigst und deine Unterschrift darf ebenfalls nicht fehlen.


\section{Wie benennt man die ausgefüllte Datei?}
Habt ihr den Antrag ausgefüllt gibt es nur noch eine Kleinigkeit die gemacht werden muss. 
Die Datei muss nach einem bestimmten Schema benannt werden. 
\[\texttt{Antrag\_Referat\_JJJJ\_MM\_TT}\]
Das Wort \texttt{Antrag} bleibt bestehen. Bei \texttt{Referat} schreibt ihr bitte eure Referatsstelle hin, aus der der Antrag kommt.
Bei \texttt{JJJJ\_MM\_TT} schreibt ihr bitte das Datum auf, an dem ihr den Antrag abgebt. Als Beispiel \texttt{2022\_04\_22}.

Somit ergibt sich ein Beispielname der lautet \texttt{Antrag\_Umwelt\_2022\_04\_22}.


\section{Wohin mit dem ausgefüllten Antrag?}
Sobald ihr den Antrag ausgefüllt habt und richtig benannt habt, schickt ihr in entweder an die Vorstände allgemein (\href{mailto:vorstand@asta-hsrm.de}{vorstand@asta-hsrm.de}). 
Andere Möglichkeit ist, die Datei in der Cloud hoch zu laden (\href{https://cloud.asta-hsrm.de/s/gxiGrnG2bTSQfBG}{cloud.asta-hsrm.de/s/gxiGrnG2bTSQfBG}).

Wir empfehlen dir, dass du beides machst, somit ist sicher gestellt das die Datei auf jeden Fall vorliegt. 

